\chapter{Introduction}


\section{Motivation}

In the research and development of generative AI-models, the computing speed and energy efficiency
are increasingly becoming the center of attention.\footcite[cf.][1]{luccioniPowerHungryProcessing2023}
The authors of Open AI confirm, that the growth rate of machine learning models 
surpassed the growth of efficiency within computerchips.
The required computing power of the models double each 3-4 months but the power of computerchips, after Moore’s Law 
double only every 2 years.\footcite[cf.][1]{darioamodeiAICompute}
Focusing on current problems like rising energy consumption of datacenters and the associated
greenhouse gas emissions, the search for more efficient solutions is essential for the future.
Worldwide energy consumption of data centers increases annually by approximately 20-40\%.\footcite[cf.][1]{hintemannDataCenters20212022} 
The International Energy Agency anticipates that by 2026, global energy consumption will double to 6\%, driven by factors such as AI, cryptocurrencies, and digitization.\footnote{cf.\cite{anon.Electricity2024Analysis2024}, p. 31-37; cf.\cite{jacksonAIBoomWill2024}, p. 1}
However, the specific contribution of AI to these total numbers remains unclear.

\section{Problem statement}

A well-known, more energy-efficient approach for AI workloads involves the use of AI accelerators based on \ac{ASIC}s.
These circuits are designed specifically for certain applications, such as Google's \ac{TPU}s.\footcite[cf.][39]{wittpahlKuenstlicheIntelligenzTechnologie2019} 
This is useful because the usage of multimodels for discirminating tasks compared to
task specific models are more energy intense.\footcite[cf.][5]{luccioniPowerHungryProcessing2023}
One promissing alternative concept in research is the usage on physics-inspired hardware accelerators, that are primarly used for optimization problems
because of their ability to solve problems faster and more efficient than GPUs.\footcite[cf.][1]{mohseniIsingMachinesHardware2022}
A scalable physics-inspired hardware accelerator (also called Ising-machine),
that surpasses the power of existing standard digital computers, could have a large influence
on practical applications for a variety of optimization problems.\footcite[cf.][1]{mohseniIsingMachinesHardware2022}

Such physics-inspired hardware accelerators offer, due to their special calculation method,
potential for efficient processing of computationally intensive tasks. 
Specifically, the acceleration in contrast to digital computers is achieved by calculating
the computationally intense tasks with analog signals.
On top of that, the implementation on dedicated hardware offers the possibility to exlpoit the parallelization
of digitwal hardware accelerators and analog computation.\footcite[cf.][4]{mohseniIsingMachinesHardware2022}

Interesting enough, despite their different applications, the energy function of the hardware accelerator that is used
in Ising-machines shows big parallels to those used in \ac{BM}s, therefore it can be suggested that Ising-machines could work well for AI.\footcite[cf.][10]{caiHarnessingIntrinsicNoise2019}
Ising-machines strive to minimize their energy, which is defined by the pairwise 
interaction of binary variables (Spins).\footcite[cf.][1]{wangOscillatorbasedIsingMachine2017} 
In contrast, \ac{BM}s are energy-based neuronal networks that are used for classification tasks by 
acllocating a skalar energy for each configuration of variables.
Minimizing the total network energy is therefore equal with the solution of a optimization problem.\footcite[cf.][2]{nazmbojnordiMemristiveBoltzmannMachine2016} 
Current problems with \ac{BM}s are the high complexity and high requirements for the all-to-allcommunication between the processing units,
which causes the implemention on conventional digital computers to be inefficient, but also 
an inherently slow convergence in certain processes such as simulated annealing.\footcite[cf.][1]{nazmbojnordiMemristiveBoltzmannMachine2016} 
These challenges complicate the training and the usage of \ac{BM}s, especially for large data volumes and complex optimization tasksl.\footcite[cf.][2]{nazmbojnordiMemristiveBoltzmannMachine2016} 
Nevertheless the similarities of both models implicate, that Ising-machines could be able to execute this 
specific AI-model with higher energy efficiency and with higher computing speed.
Currently there are only few concepts that exists on how to achieve a implementation of a \ac{BM} on a Ising-machine.
The paper of the authors Mahdi, Nazm, BojnordiEngin and Engin Ipek developed promissing proof of concept.
However it could not be shown how a implementation on a real accelerator chip could function.

With the given background, the following central research question and two sub-questions arise for this thesis:
\begin{enumerate}
    \item Can Boltzmann Machines be efficiently implemented on a physics-inspired Hardware accelerator by analog noise injection?
        \begin{itemize}
            \item What is the accuracy of the AI-model on the hardware accelerator compared to conventional methods in terms of efficiency and accuracy?
                \begin{itemize}
                    \item Metrics: Prediction accuracy and negative Likelihood
                \end{itemize}
            \item How is the accelerators performance in terms of computing speed and energy efficiency compared with other hardware accelerators in literature?
                \begin{itemize}
                    \item Metrics: Throughput (Samples/Sec), Energy consumption (Energy/Operation)
                \end{itemize}
        \end{itemize}  
\end{enumerate}

It is therefore necessary to test whether this generative AI model is compatible with Ising machines
machines and whether the solution is efficient or not.


\section{Objective}

The primary objective of this bachelor thesis is the research and extension of a existing physics-inspired
hardware accelerator(Ising machine) for the implementation and evaluation of \ac{BM}s as an energy based 
AI-model. The aim is to answer the posed research question. 
In addition to that, it would be beneficial if rules for the influence of hyperparameters could be established
since there is no data available for this new method.

To initially accomplish this objective it is necessary to establish a simulator pipeline to the
hardware accelerator that translates \ac{BM} on top of it.
The simulator pipeline consists of an existing machine learning library and an existing hardware accelerator
that need to be connected to each other.
First the simulator pipeline needs to verify that it is possible for the hardware accelerator
to realize \ac{BM}s. 
Within the simulator pipeline, the activation probabilities of individual neurons are measured on the simulated hardware.
If this process proves successful, it is then expanded to simulate a complete neuronal network.
The final step is that the hardware accelerator can be used for training and iterference
and is comparable to conventional machine learning libraries.
This phase includes a carefully adjustment and possibly extension of the existing accelerator to be compliant 
with the specific requirements of \ac{BM}s.

If the simulator pipeline can be validated a workload consisting of a standard data set
to recognize handwritten digits is tested on it.
The prediction accuracy and negative likelihood are investigated to answer the first research question.
In a next step, the throughput (samples/sec) and energy consumption (energy/operation) of the \ac{BM} on the Ising hardware accelerator is to be collected.
These metrics aim to address the second part of the research questions posed.

\section{Research method}

The applied research methodology for this thesis is \ac{DSR} by Österle et al..\footnote{cf.\cite{oesterleMemorandumZurGestaltungsorientierten2010}, p.1-6; cf.\cite{oesterleKonsortialforschung2010}, p. 273-274}
\ac{DSR} is chosen because it supports the generation of new knowledge and ensures that research results are both theoretically and practically
applicable, while they also are in line with the objectives of the project.
Furthermore, this allows to find a solution for practical problems through the iterative approach adding more functionalities over time but also leaving room for fixing errors and general improvements.
In the initial design phase, an overall solution architecture is modeled, from which the requirements and functionalities for implementation are derived.
Within the iterative design and evaluation phases of the \ac{DSR} framework prototyping is used to exploratively as a fast and explorative way to implement the simulator pipeline according to G. Arthur Mihram's prototyping model.\footcite[cf.][71-72]{mihramSimulationMethodology1976}
Despite, in the last general evaluation phase, the method of simulation is used, since at this time the prototype includes all required functionalities.
The aim of the simulation is to meassure the performance of the solution and to make the hardware accelerator comparable. 

\section{Structure of the thesis}
The thesis has six chapters that are primarly structured by the guidelines given by Holzweißig.\footcite[cf.][32-40]{holzweissigWissenschaftlichesArbeiten2017}
The first chapter is the ``introduction'' focussing on the relevance and motivation for the new application of more sustainable AI-models
with the novel solution of the physics-inspired hardware accelerator and what research questions should be answered.
In the second chapter, the ``current state of research and practice'' is discussed with 
existing concepts but moreover all the required concepts that the hardware accelerator utilizes are explained 
and set into perspective to lay the foundation for the ongoing practical implementation part of this thesis.
Next, the third chapter explains the applied research methodologies utilized in this thesis and why the decision is to use them.
The fourth chapter covers the implementation of the Simulator pipeline and is strucuted after the \ac{DSR} phases.
It ends with a finished prototype, which the performance is meassured and therefore chapter five focusses on the 
diffusion and discussion of the results. 
Lastly, chapter six aims to give an overall critical reflexion and outlook on the thesis's applied methodology and achieved results.









