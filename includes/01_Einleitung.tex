\chapter{Einleitung}


\section{Motivation}
\section{Problemstellung}

In der Forschung und Entwicklung von Generativen KI-Modellen rückt die
Rechengeschwindigkeit und Energieeffizienz zunehmend in den Fokus\footcite[Vgl.][1]{luccioniPowerHungryProcessing2023}
Die Autor*innen von Open AI bestätigen, dass die Wachstumsrate von
Machine-Learning-Modellen die Effizienzrate von Computerchips schon längst
übertroffen hat. So verdoppeln sich jede 3-4 Monate der Rechenbedarf dieser
Modelle jedoch verdoppeln sich nach Moore’s Law die Leistung der Computerchips
nur jede 2 Jahre.\footcite[Vgl.][1]{darioamodeiAICompute}
Angesichts der Probleme des steigenden Energieverbrauchs von Rechenzentren
und den damit verbundenen Treibhausgasemissionen dieser, ist die Suche nach
effizienteren Lösungen essenziell für die Zukunft. Weltweit steigern Datenzentren
ihren Energieverbrauch jährlich um 20-40\%, wodurch sie 2022 etwa 1,3\% des
globalen Energieverbrauchs und 1\% der energiebedingten globalen
Treibhausgasemissionen verursacht haben.\footcite[Vgl.][1]{hintemannDataCenters20212022} 
Jedoch ist hier nicht zu erkennen, wie groß dabei der KI-Anteil zur Grundgesamtheit beiträgt.

Ein bereits bekannter Ansatz ist die Benutzung von KI-Beschleunigern basierend auf
ASICs (Application-specific Integrated Circuits) - also Schaltungen, die
anwendungsspezifisch verwendet werden, wie zum Beispiel Google TPUs (Tensor
Processing Unit).\footcite[Vgl.][39]{wittpahlKuenstlicheIntelligenzTechnologie2019} 
Dies ist auch sinnvoll, da die Verwendung von Mehrzweckmodellen für diskriminierende Aufgaben im Vergleich zu
aufgabenspezifischen Modellen energieintensiver ist.\footcite[Vgl.][5]{luccioniPowerHungryProcessing2023}
Ein alternatives vielversprechendes Konzept in der Forschung ist die Verwendung von
physikinspirierten Hardwarebeschleunigern, die primär bei Optimierungsalgorithmen
eingesetzt werden aufgrund ihrer Fähigkeit Probleme schneller und effizienter als
GPUs lösen zu können.\footcite[Vgl.][1]{mohseniIsingMachinesHardware2022}
Ein skalierbarer physikinspirierter Hardwarebeschleuniger (auch Ising-Maschine
genannt), der die Leistung bestehender Standard-Digitalrechner übertrifft, könnte
einen großen Einfluss auf praktische Anwendungen für eine Vielzahl von
Optimierungsproblemen haben.\footcite[Vgl.][1]{mohseniIsingMachinesHardware2022}

Solche physikinspirierten Hardwarebeschleuniger bieten durch ihre besondere
Berechnungsweise Potenzial für eine effizientere Verarbeitung von
rechenintensiven Aufgaben. Konkret wird die Beschleunigung, anders als es bei
digitalen Computern der Fall ist, durch die Berechnung rechenintensiver Aufgaben
mit analogen Signalen erreicht. Die Implementierung auf dedizierter Hardware bietet
darüber hinaus die Möglichkeit, die Parallelisierung von digitalen
Hardwarebeschleunigern und analogem Rechnen auszunutzen.\footcite[Vgl.][4]{mohseniIsingMachinesHardware2022}

Interessanterweise zeigen die Energiefunktionen von Hardwarebeschleunigern, die
in Ising-Maschinen verwendet werden, große Parallelen zu denen in Boltzmann
Maschinen, trotz ihrer unterschiedlichen Anwendungen, daher liegt es nahe, dass
Ising Maschinen auch für KI gut funktionieren.\footcite[Vgl.][10]{caiHarnessingIntrinsicNoise2019}
Ising-Maschinen zielen darauf ab, ihre Energie zu minimieren, wobei sie Energie als eine paarweise
Interaktion von binären Variablen „Spins“ definieren.\footcite[Vgl.][1]{wangOscillatorbasedIsingMachine2017} 
BoltzmannMaschinen hingegen sind energiebasierte neuronale Netzwerke, die
Klassifizierungen durchführen, indem sie jeder Konfiguration der Variablen eine
skalare Energie zuordnen. Die Netzwerkenergie zu minimieren ist hierbei
vergleichbar mit der Lösung des Optimierungsproblems. \footcite[Vgl.][2]{nazmbojnordiMemristiveBoltzmannMachine2016} 
Aktuelle Probleme mit Boltzmann-Maschinen umfassen die hohe Komplexität und
Anforderungen an die All-to-All-Kommunikation zwischen Verarbeitungseinheiten,
was ihre Implementierung auf herkömmlichen digitalen Computern ineffizient macht,
sowie eine inhärent langsame Konvergenz in bestimmten Prozessen wie Simulated
Annealing.\footcite[Vgl.][1]{nazmbojnordiMemristiveBoltzmannMachine2016} 
Diese Herausforderungen erschweren das Training und die Anwendung von Boltzmann-Maschinen insbesondere für große Datenmengen
und komplexe Optimierungsaufgaben.\footcite[Vgl.][2]{nazmbojnordiMemristiveBoltzmannMachine2016} 
Nichtsdestotrotz impliziert die Ähnlichkeit der beiden, dass Ising-Maschinen in der Lage sein könnten, dieses
spezielle KI-Modell, energieeffizienter und mit höherer Rechengeschwindigkeit
auszuführen. Aktuell existieren nur wenige Konzepte eine Implementierung von
Boltzmann Maschinen auf Ising-Maschinen zu erreichen. Das Paper der Autoren Mahdi Nazm BojnordiEngin und Engin Ipek ist hier ein
vielversprechender Ansatz, jedoch konnte nicht gezeigt werden, wie es auf einem
richtigen Beschleunigerchip funktionieren würde.

Vor diesem Hintergrund ergeben sich folgende zentrale Forschungsfragen:

\begin{enumerate}
    \item {Können Boltzmann Maschinen auf physikinspiriertenHardwarebeschleunigern durch analoge Rauschinjektion effizient implementiert werden?}
        \begin{itemize}
            \item Wie ist die Genauigkeit des KI-Modells im Hardwarebeschleuniger? Metrik: Prediction Accuracy

            \item ergleichen mit anderen Hardwarebeschleuniger, FPGA, GPU oderCPU aus der Literatur (gute und schlechte) in Bezug auf Energieeffizienz und Rechengeschwindigkeit – Metriken: Troughput(Samples/Sec), Energieverbrauch (Energy/Operation)
        \end{itemize}  
\end{enumerate}

Daher gilt es zu testen, ob dieses generative KI-Modell mit Ising Maschinen
kompatibel ist und ob diese Lösung effizient ist oder nicht.


\section{Zielsetzung(ohne gneaue Metriken nennen, generell halten)}

Das primäre Ziel dieser Bachelorarbeit ist die Erforschung und Erweiterung eines
bestehenden physikinspirierten Hardwarebeschleunigers (ISING Maschine) zur
Implementierung und Evaluation von Boltzmann Maschinen, einem
energiebasierten KI-Modell. Dabei sollen die aufgestellten Forschungsfragen
beantwortet werden.

Hierzu ist es zu Beginn nötig eine Simulator Pipeline zu konstruieren mit der
Boltzmann Maschinen auf dem Hardwarebeschleuniger übersetzt werden. Die
Simulator Pipeline besteht dabei aus einer bestehender KI-Bibliothek und
bestehenden Hardwarebeschleuniger, die miteinander verbunden werden. Mit der
Simulator Pipeline soll gezeigt werden, dass der Hardwaresimulator die Boltzmann
Maschinen umsetzen kann. Aus der Simulator Pipeline heraus werden die
Aktivierungswahrscheinlichkeiten der einzelnen Neuronen auf der simulierten
Hardware gemessen und bei Erfolg bis zu einem vollständigen Neuronalen
Netzwerk erweitert. Finaler Schritt ist, dass der Hardwarebeschleuniger für Training
und Interferenz genutzt werden kann und dabei vergleichbar mit herkömmlichen MLLibraries ist. Diese Phase umfasst die sorgfältige Anpassung und möglicherweise
Erweiterung des bestehenden Beschleunigers, um die spezifischen Anforderungen
der Boltzmann Maschinen zu erfüllen.

Wenn die Simulator Pipeline validiert werden kann, wird ein Workload auf ein
Standarddatenset zur Handschrifterkennung getestet. Dabei werden die Prediction
Accuracy, Troughput (Samples/Sec) und der Energieverbrauch (Energy/Operation)
der Boltzmann Maschinen auf dem ISING Hardwarebeschleuniger untersucht und
dadurch die aufgestellten Forschungsfragen beantwortet.

\section{Forschungsmethodik}

Design Science Research

\begin{enumerate}
    \item \textbf{Problemorientierung:} DSR fokussiert auf die Lösung praktischer Probleme, wie die Forschung zur Steigerung der Effizienz und Rechengeschwindigkeit in KI-Modellen.
    \item \textbf{Artefakt Entwicklung:} Zentral in DSR ist die Entwicklung innovativer Artefakte. Die Arbeit zielt darauf ab, ein solches Artefakt in Form des physikinspirierten Hardwarebeschleunigers weiterzuentwickeln und für KI-Modelle einzusetzen.
    \item \textbf{Iterative Evaluation:} Durch die iterative Vorgehensweise in DSR kann die Ausarbeitung der Lösung fortlaufend verbessert und angepasst werden, was für die Entwicklung und Optimierung von KI-Systemen entscheidend ist (ebenfalls das Konzept).
    \item \textbf{Beitrag zur Wissensbasis und Praxisrelevanz:} DSR unterstützt die Generierung neuer Erkenntnisse und stellt sicher, dass Forschungsergebnisse sowohl theoretisch fundiert als auch praktisch anwendbar sind, was mit den Zielen Ihres Projekts im Einklang steht. Untermethodik könnte hierbei eine Simulation sein. Variabel, je nach Verlauf der Forschung.
\end{enumerate}


\section{Aufbau der Arbeit}