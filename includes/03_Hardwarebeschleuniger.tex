\chapter{Objective specification and presentation of the research methodology}


Having laid the groundwork with essential concepts necessary for this thesis,
this chapter aims to outline the objectives of the practical segment as well as the research methodology employed to achieve them.
In the first part, the specific objectives are defined. 
Afterwards, in a second part the used reasearch methods Design Science Research and Prototyping are explained.






\section{Objective specification (genauer als in Einleitung, Metriken erwähnen, Erfolg meiner Methode bewerten, Welcher Teil der Forschungsfrage wird beantwortet?)}

In contrast to the mentioned papers in 2.4.4, this thesis wants to use their proof of concepts as a basis and go further to see if the concept
is feasible on the complete \ac{ASIC} hardware accelerator like the \ac{mem-HNN} and does it bring an actual acceleration.
The practical part is therefore using a current \ac{ASIC} with not only the memristor crossbar array but also all the other requried hardware components.
Furthermore, the thesis focuses on an actual training of a \ac{RBM} and also the interference of it to be able to answer the research question.
Expanding upon the foundational work, this research explores the implementation of the N/2 synchronous update mechanism.
This design choice emphasizes a expactation of higher sampling speeds and efficiency.

At the beginning, the \ac{IT} artifact to be implemented is modeled and all components, transitions and processes of the overall solution are identified.
As a result of the implementation a complete performance evaluation, including a comprehensive energy model and a latency model should be established.
This evaluation aims to measure the processing time required for data and the energy consumed.
The results should be compared to other sampling methods, such as Gibbs and Metropolis sampling.
Lastly, this thesis performs hyperparameter tuning in oder to gather new data on how these machines should be configured for optimal performance.
Since, there is no data for hyperparameter tuning of such a concept in the literature yet this establishes foundational work on how to make use of artificial intelligence within such an accelerator.
Through this process, the research seeks to determine the feasibility of a proper training with this setup, focusing on its practicality and efficiency.
By benchmarking these aspects against traditional sampling methods, the thesis aims to underscore the potential of the \ac{mem-HNN} 
in practical training of \ac{RBM}s.

\section{Design Science Research }
\section{Zielsetzung(ohne gneaue Metriken nennen, generell halten)}
\section{Prototyping / Simulation}

