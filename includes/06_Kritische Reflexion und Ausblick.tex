\chapter{Critical reflexion and outlook}
This chapter concludes the research work by summarizing the key findings and provide a critical reflection on these outcomes.
The goal is to retrospectively evaluate the core structure and methodology of this study, discussing how effectively it addressed the posed research question.
Additionally, this section aims to illustrate the practical and theoretical consequences of the artifacts and insights developed through this work.
By doing so, it underscores the relevance of the study’s contributions to both academic research and real-world applications.
Lastly, this chpaer offers an outlook on future research, improvements and implications in general.

\section{Mission of the work}
The general mission of the work is the implementation of Boltzmann Machines on a physics-inspired 
hardware accelerator by analog noise injection. 
Especially, the possibility of modelling the Boltzmann Machines and afterwards meassuring the performance are the focus of the implementation.
This research is motivated by ever increasing energy consumption for artificial intelligence models and stagnating computing speeds.
Therefore, the currently developed \ac{mem-HNN} \ac{ASIC} hardware accelerator, which HPE developed and researches on 
should be further researched for the new aplication of efficiently modelling this neural network.

This thesis first explains all the relevant fields in chapter two.
Here, topics like Neural Networks and sustainability is handled and within the concepts of energy-based models,
Boltzmann Machines and \ac{RBM}s are processed. 
After discussing neural networks, the focus shifts to the hardware accelerator, providing a comparison of current developments and research that lead to the functioning of the targeted \ac{mem-HNN} accelerator.
Hence, the concept of Hopfield Networks that are inspired out of physics, and technical parts like the crossbars are explained to 
finally bring the idea of noise injection to the reader. 
This idea makes it possible to train and interfer the Boltzmann Machines on the \ac{ASIC} accelerator.

The theoretical research theory is to be implemented in the practical part of the thesis.
This is done by following the scientific \ac{DSR} process by Österle et al. to create an \ac{IT}-artifact,
which is developed iteratively and consists out of three design and evaluation phases. 
The design phase itself follows the methodology of a prototyping implementation by G.A. Mihram that allows rapid and explorative proceed.
In the beginning the prototyping established a function catalog and divided the single functions, 
which simulate the functions of the developed overall solution, into the single design phases. 
The focus lays on using the Hopfield Network as sampling method to train the Boltzmann Machine
and therefore show the possibility of the realizability on the physics-inspirated \ac{mem-HNN} accelerator.
The implementation of the prototyp matches well with the \ac{DSR} framework and is done icrementally to develop the prototyp further and use the previous output as
the new input for the next iterative phase.   
The last evaluation phase uses the finished prototype with desirable all functionalities as input and 
aims at meassuring the performance of the successful implementation of the Boltzmann Machines compared to the earlier established intrinsic baselines
of the conventialnal approaches Gibbs sampling and Metropolis Hastings.
Specifically, this is done by performing the methodology simulation by Kellner/Madachy/Raffo and setting the focus on 
the metrics throughput and energy consumption. 
All the single methodologies were chosen to fit well into the iterative \ac{DSR} framework in order to best 
create the desired \ac{IT}-artifact with a high scientific procedure.
The desired result is a functioning representation of the \ac{mem-HNN} with all the functionalities included, 
evaluated and meassured performance with additional information about the Hyperparamter tuning.

\section{Critical reflection on the results and methodology}

This thesis represents a significant advancement in the field of energy-efficient AI models by introducing the first-ever simulator pipeline designed for Boltzmann Machines.
The existing proof of concenpts are taken further and the research question is answered.
Furthermore, an intrinsic comparison between the \ac{mem-HNN} \ac{ASIC} and the conventional methods of Gibbs Sampling and Metropolis Hastings is established.
The results of this thesis, while promising, warrant a thoughtful examination in terms of applied methodologies and meassured results.

\textbf{Reflection on Results:} The introduction of the simulator pipeline has not only demonstrated the possibility 
of the implementation of Boltzmann Machines with a Hopfield Network as sampling method but also the results regarding
in modeling energy efficiency and computational throughput in the Simulator Pipeline is very positive.
Therefore, the overall posed research question is successfully answered and results in a tool that has never been explored in previous studies but also
provides a new tool that can be adapted for different workloads in future research. 
In an second part looking at meassuring the energy efficiency, especially the energy based model has been partially addressed.
This is because the metric energy consumption does not include all hardware compontents, already mentioned in chapter 4.6.
Hence, the energy usage of the bus system to the digital computer, the controller and the memory on the \ac{mem-HNN} 
are not currently not included.
The digital computer that updates the weights would require the highest energy consumption is also not included due to missing energy values.
An more correct energy consumption of the overall solution architecture is part of future work with approximately two years develoment time.
Therefore the current energy model for the \ac{mem-HNN} is considered a first attempt but with the Simulator Pipeline
the energy model can be extended trouble-free for example a bus system, controller or digital computer.
Still before the Hopfield Network could be used as sampling method, conventional options like \ac{MCMC} 
require higher amounts of sampling iterations, and even if mapped on accelerators consume around 15x more energy.
When looking at another paper with a complete energy model comparing electronical with analog hardware, the energy required is 
higher than this thesis results but the analog solution with our missing part only increases the energy consumption by 30\% and in comparison to
the digital solution in total 5x more energy efficient.\footcite[cf.][12-13]{demirkiranElectroPhotonicSystemAccelerating2023}

Furthermore, the resolution of the weights are set perfect in this thesis, which does not represent the
exact representation of the \ac{mem-HNN}. In the energy model the resolution is considered but
not for the results of the prediciton accuracy. 
Despite this, many indicators have performed well even with low bit resolution suggesting that even minimalistic settings can yield substantial results.\footnote{cf.\cite{maEra1bitLLMs2024}, p. 1; cf.\cite{GitHubHtqinQuantSR}, p. 1; cf.\cite{rouhaniMicroscalingDataFormats2023}, p. 1; cf.\cite{rouhaniSharedMicroexponentsLittle2023}, p. 1}
Lastly, the \ac{RBM} was chosen as simpler AI-model to collect meaningful baselines but also general Boltzmann Machines can be implemented on more complex workloads. 

\textbf{Reflection on Methodology:} 
Utilizing \ac{DSR} as the applied research methodology facilitates the incremental and continuous progress of the Simulator Pipeline as \ac{IT}-artifact.
This approach is well-suited for the implementation of the novel Simulator Pipeline especially since it requires agility.
Despite the result being specific for the \ac{mem-HNN} the \ac{DSR} procedure can be generalized for other AI-models or other data sets on different \ac{ASIC} or 
even \ac{FPGA} accelerators. 
The protyping methodology enables a rapid and explorative implementation and therefore as well suits the thesis's goal. 
Although, since performance is no criteria of prototyping and the training of the model is performed on a notebook with a CPU\footnote{\texttt{Intel i7-10610U, 1.80GHz, 2304 Mhz, 4 Core(s), 8 Logical Processor(s)}} that takes around 30-40min. per training,
optimization can be seen as critics and furhter implementation goal.
However, the simulation only estimates the actual hardwares capabilities.
Once the hardware is available real-world results can be different making it essential to build and validate a physical hardware accelerator based on these initial analyses.
These difference can be energy effciency, throughput and prediction accuracy. 
Still, the functionalities are modelled with great effort and are an indication on how the Pipeline behaves and already delivered close to real-world results in another paper.

\section{Extrusion of results for theory and practice}
The resulting Simulator Pipeline for the \ac{mem-HNN} established in this thesis achieved 
promising results of relevance for the scientific community and practical applications. 
The intrinsic comparison with other analog probabilistic accelerators revealed that the \ac{mem-HNN} offers competitive throughput while enhancing energy efficiency,
thereby highlighting its potential as a transformative tool in the realm of energy-efficient computing.
Also, a first estimation is done by reviewing other analog probalistic accelerators and comparing the energy efficiency and throughput which result in promising and competitive estimates. 
Additionally, as the Simlator Pipeline is generalizable, in the furure different workloads can be chosen as starting point for research with AI-models 
for Boltzmann Machines. 
Hence, even greater improvements could be achieved for these worklaods.
Of course the results themselves as part of the intrinsic comparison deliver important insights and close a research gap of this novel sampling method 
and give confidence in the Solution, hyperparameters tuning and perforance. 
As the tool was developed using a scientific method, it is easy to track why a program section, for example, has certain properties and ensures fewer errors after iteratively correcting them.
For further research within HPE the results validate that training for Boltzmann Machines is possible and that in 
a next step this it is planned to research further with other workloads, architectures and changing from \ac{RBM} to a general Boltzmann Machine.
Furthermore, the current plan is to collect the result of this thesis and include them into a paper 
proposed and submitted to TechCon, which is an HPE internal research fair laying the ground for HPE's future investments.
Lastly, the results are also planned to be presented at public fairs HP Labs is visiting as presenter. 

\section{Outlook}
The outlook for the \ac{mem-HNN} Simulator Pipeline is the improvement of missing functionalities to achieve better and more correct results. 
Here topics like the bit resolution of the \ac{ASIC} is planned to be integrated but at the same time the energy model is planned
to be modified in further research to gather a nearly real-world energy consumption of the Simulator Pipeline.
The overall desired outlook is to test and verify more complex worklaods on the tool and as an underlying model use Boltzmann Machines 
instead of the simpler form with a \ac{RBM}.
Furthermore, there are suggestions for future work to improve the generalizability and scalability of the solution architecture to make Boltzmann Machines scalable.
