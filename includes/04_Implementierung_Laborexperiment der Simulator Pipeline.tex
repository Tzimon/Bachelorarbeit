\chapter{Implementierung/Laborexperiment der Simulator Pipeline}

Hopfield Netzwerk aktivierungsfunktion der Updating methode

-> Konzeptionell Art des Updates mit keiner Temperatur wie bei MCMC 
Unterschied von MCMC zu Hopfield Netzwerk -> Zufällige Konfiguration und minimale Energie finden. Jedoch hat ein Hopfield
Netzwerk keine Temperatur 

-> Starte zufällige Konfiguration
-> Wähle ein Neuron aus und Berechne Summe und addiere mit Bias, 
-> Update wenn thresshold überschritten 1 und dann auf 0 
-> Speichern der neuen Konfuguration 
-> Starte iteration von gespeicherter Konfiguration 
-> Am Ende habe ich 10000 Vektoren (Die Konfigurationen) -> V1 Neuron wurde so und so oft aktiviert und ich muss average
über das neuron und habe dadurch die Aktivierungswarscheinlichkeit.

-Aktivierungsfunktion einfügen (Binary Step und verfleich zu sigmoid von Abb.4)


-Testen der Aktivierungsfunktion, wenn ich ein Neuron trainiere und dann Mitteln 
- Von vornerein auf Netzwerk Basis arbeiten mit mehren Neuron, jedoch für 1 Neuron testen




\section{Zielsetzung und Forschungsmethodik}
\section{Aufbau der Simulator Pipeline}
\section{KI-Bibliothek Scikit-Learn}
