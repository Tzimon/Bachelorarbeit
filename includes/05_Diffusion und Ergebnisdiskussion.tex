\chapter{Diffusion and discussion of results}

von Holzweißig: Dieses dient dazu die Ergebnisse des eigenen Beitrags zusammenzufassen
und kritisch zu diskutieren. Auch kann hier der Versuch einer Ergebnisverallgemeinerung
erfolgen. Vergleich mit der Literatur
Evaluation der Erkenntnisse in Bezug auf die Zielsetzung intrinsisch

Was sind miene Hauptergebnisse? 
\section{Solution architecture and Hyperparameter tuning}
\subsection{Gibbs sampling}
evtl. weglassen und nur als Baseline benutzen
\subsection{Metropolis Hastings}
evtl. weglassen und nur als Baseline benutzen
\subsection{Single update Hopfield Network}
Hyperoarameter Scale und Iterations
\subsection{N/2 Half update Hopfield Network}
Hyperparameter Scale und Iterations
\section{Throughput}
Vergleich mit CPU usw. für N/2 Half
\section{Energy consumption}
Vergleich mit CPU usw. für N/2 Half
\section{diffusion}
Inspiration, kann ähnlich sein
Veröffentlichung in den Labs, Paper veröffnetlichung, auf messen gehen mit Ergebnissen,
Weiterentwicklung für Komplette umsetzung des Modells auf dem richtigen hardwarebeschleuniger,
und Einreichung der Bachelorarbeit an die DHBW
