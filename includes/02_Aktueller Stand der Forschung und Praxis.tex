\chapter{Aktueller Stand der Forschung und Praxis (generell auch wiedergeben von aktuell existierenden Lösungsmustern)}

\section{Ressourcenverbrauch bei KI-Modellen}
\subsection{Ressourcenverbrauch bei KI-Modellen}
\subsubsection{Nachhaltigkeit}
\subsubsection{Stromverbrauch}
\subsubsection{Rechenleistung begrenzt, KI-Modelle wachsen schneller als verfügbare Leistung}

\section{Deep Neural Network - Boltzmann Maschinen (Erstmal DNN erklären generell)}
\subsection{Konzept und Anwendung des Modells }
\subsection{Aktuelle Probleme mit RBM/BM}
\subsection{Energiefunktion}
\subsection{Training von BMs}
\subsubsection{Markov-Chain-Monte-Carlo-Verfahren}
Metropolis Hastings

Conrtrastive Divergence

\section{Hardwarebeschleuniger}
\subsection{Aktuelle Ansätze im Bereich KI und weitere Lösungen}
\subsubsection{Asics}
\subsubsection{Quantencomputing}
\subsection{ISING Maschine/ Physikinspirierter Hardwarebeschleuniger}
\subsubsection{Konzept (mit Energiefunktion), Probleme der Digitalrechner bzw. Unterschied zu Digitalrechner}
\subsubsection{Aktuelle Anwendung}
\subsubsection{Potentielle Einsatzgebiete für KI-Modelle}
\subsubsection{Parallelen Energiefunktion BM und ISING Maschine}

\section{Memristor Hopfield Network}
\subsection{Memristor}
\subsection{Hopfield Network}
\subsection{Crossbar}
\subsection{Output Hopfield Networtk}
\subsection{Noisy HNN}
